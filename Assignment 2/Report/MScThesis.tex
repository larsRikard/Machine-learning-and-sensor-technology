%% sample template file for a MSc Thesis
%% The default is with two sided setup:
\documentclass[%
oneside,    %% uncomment for onesided layout
project,    %% uncomment not thesis but project report
nosummary   %% uncomment if no summary page should be generated
]{USN-MSc}

% The following command removes the chapter names form the header
% (comment/remove) if you prefer to have them:
\pagestyle{plain}

% --- Bibliography setup ---
%%% default is the "ieee" style
\usepackage[style=ieee, sorting=none]{biblatex}
%%% If you want to use "author-year" style
%%% where `\cite{Foo2011}` generates "Foo et al. (2011)"
%%% and   `\parencite{Foo2011}` generates "(Foo et al. 2011)"
%%% then comment the line above and use
%\usepackage[style=authoryear]{biblatex}
%%% or
%%% if you want to use "alphabetic" style then use
%%% where `cite[Foo2011]` generates "[Foo11]"
%%% then comment the line above and use
%\usepackage[style=alphabetic]{biblatex}
%%% instead.
%% load the bib file:
\addbibresource{MscThesis.bib}

\usepackage{lipsum} % just for providing fill text used in this template
\usepackage{array} % for adjusting tables?

% --- general setup ---
%% Please fill in the following parameters:
\newcommand{\mytitle}{%
%% title:
Assignment 2 - Data Visualization and Regression Models
}

\newcommand{\mysubtitle}{%
%% master programme (for thesis only)
%% uncomment the appropriate one:
%Electrical Power Engineering
%Energy and Environmental Technology
Industrial IT and Automation
%Process Technology
}

\newcommand{\mykeywords}{%
%% keywords (for thesis only):
<keyword one, keyword two, \ldots>
}

\newcommand{\myauthor}{%
%% author(thesis) or group code (project):
Lars Rikard Rådstoga
}

\newcommand{\myparticipants}{
%% group participants (for project only)
Lars Rikard Rådstoga
}

\newcommand{\supervisor}{%
%% supervisor:
<Supervisor's Name>}

\begin{document}

% --- title page setup ---
\USNtitlepage%
%% Please provide the following information:
%% #1 optional figure (set to {} if not wanted)
{%
  {\normalsize <optional figure>}
  \includegraphics[draft,width=\textwidth]{USN_logo_en}}
%% #2 Project partner:
{<Project partner>}
%% #3 Summary:
{%
  \lipsum[6-7]
}

%\chapter*{Preface}
%\label{ch:preface}
%\addcontentsline{toc}{chapter}{Preface}
%\lipsum[1-3]
%\bigskip
%Porsgrunn, \today
%\myauthor %% for thesis
%\myparticipants %% for project


%% table of contents
\tableofcontents
\addcontentsline{toc}{chapter}{\contentsname}

%\listoffigures % out-comment if unwanted
%\addcontentsline{toc}{section}{\listfigurename}

%\listoftables  % out-comment if unwanted
%\addcontentsline{toc}{section}{\listtablename}

%\chapter*{Nomenclature}
%\label{sec:nomenclature}
%bla

%\begin{longtable}{ll}
%  \textbf{Symbol} & \textbf{Explanation}\endhead\\
%  A/D	& Analogue-Digital-Converter \\
%  CMR	& Common Mode Rejection \\
%  foo	& Foo \\
%  bar 	& Bar
%\end{longtable}

\chapter{Understanding the problem}
\label{ch:understanding}
The following chapter discusses and answers a few questions regarding the problem and the dataset, to better understand the following chapters.

\section{What type of problem is it?}
\label{sec:typeOfProblem}
The problem at hand is in fact a regression task. The dataset available contains monitored sensory data, with noise, and can be considered labeled. The goal is to develop a model that can predict/extrapolate the RUL (Remaining Useful Life) of turbofan jet engines.

\section{What category of machine learning is required?}
\label{sec:mlCategory}
As the dataset can be considered labeled, i.e. the columns in the dataset are labeled and have some physical meaning, the supervised learning category will be used.

\section{What does each column in the dataset represent?}
\label{sec:datasetColumns}

\begin{table}[!ht]
  \caption{Column descriptions}
  \centering
  \begin{tabular}{ | m{3cm} | m{10cm} |}
    \hline
    Engine & Identification number,                                        \\ \hline
    Cycle  & percentage of rotation or counted units since initialization. \\ \hline
  \end{tabular}
  \label{tab:columnDescr}
\end{table}


wtfff

\begin{figure}[!ht]
  \centering
  \includegraphics[width=1\textwidth]{Plotted data}
  \caption{
    A plot of the data given in the assignment.
    The first plot shows both the measured and the simulated acceleration data.
    And the second shows the difference between the two.}
  \label{fig:raw_data_plot}
\end{figure}

\chapter{Data Visualization}
\label{ch:visualization}
\lipsum[1]

\chapter{Preprocessing and feature selection}
\label{ch:preprocessing}
\lipsum[1]

\chapter{Regression models}
\label{ch:regression}
\lipsum[1]

\chapter{Regression models with extended features}
\label{ch:regressionExtended}
\lipsum[1]


% A dummy command that causes all bibliographyentries to be displayed
% even though there were not cited in the document. Used for demonstration
% purposes only in this template file.
~\nocite{*}

\cleardoublepage

% The bibliography should be displayed here...
%\printbibliography[heading=bibintoc]
% You rather like to call the bibliography "References"? Then use this instead:
\printbibliography[heading=bibintoc, title={References}]


\appendix
%%\renewcommand{\appendixname}{Paper} %% So we get 'Paper X' displayed instead
%
%
%\chapter[Short Title of Paper A]{Title of Paper A (probably very long and therefore not good to have in the header)}
%\label{paper-a}
%
%\paragraph{Note}
%Since some papers tend to have a rather long title it is good to provide the optional short title which then will be displayed in the table of contents and header instead of the long original title.
%On the openening page of the chapter the orginal \emph{long} title will be displayed.\bigskip
%
%\emph{Short descriptive text of paper follows here.}\bigskip
%
%The paper itself needs to be included in the published form as PDF on the next pages.
%This can be done using the \texttt{pdfpages} package by adding the command:
%
%\begin{verbatim}
%\includepdf{pages=-,openright}{Filename}
%\end{verbatim}
%
%You can omit the \texttt{.pdf} when specifying the \texttt{Filename}. Also you should include always include the option \texttt{openright} since it would look strange to have the paper starting at the back of the cover page.
%
%There are more options like only adding specific pages:
%\begin{verbatim}
%\includepdf{pages=2-6,openright}{Filename.pdf}
%\end{verbatim}
%
%For more options see Appendix~\ref{paper-b} where the most important pages of the \texttt{pdfpages} manual were inlcuded using \texttt{pdfpages}.
%
%
%%%% Command to include a PDF file directly including all pages:
%
%
\chapter[Source code]{Source code}
\label{paper-a}
%Short descriptive text of paper follows here.
The source code, Jupyter notebook, used to load, transform and plot data.
%Here we included the first five pages of the \texttt{pdfpages} manual itself.
%\includepdf[pages=1-4,openright]{fig/Railroad predictive maintenance}
\end{document}

%%% Local Variables:
%%% mode: latex
%%% TeX-master: t
%%% End:
